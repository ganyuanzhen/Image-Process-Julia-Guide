\chapter{Julia入门}\label{ch1}

\section{Julia简介}\label{sec1-1}

Julia 是一个非常棒的科学计算语言。它像 R、MATLAB 和 Python 一样简单,在高级数值计算方面有丰富的表现力,并且支持通用编程。根据官网的描述,Julia是一门这样的语言\cite{Juliadevelopers2018}:

\begin{itemize}
	\item 快速:Julia一开始就是为高性能而设计的。Julia可以通过LLVM而跨平台被编译成高效的本地代码。
	\item 通用:Julia使用多分派作为编程范式,使其更容易表达面向对象和函数式编程范式。标准库提供了异步I/O,进程控制,日志记录,性能分析,包管理器等等。
    \item 动态:Julia是动态类型的,与脚本语言类似,并且对交互式使用具有很好的支持。
    \item 数值计算:Julia擅长于数值计算,它的语法适用于数学计算,支持多种数值类型,并且支持并行计算。Julia的多分派自然适合于定义数值和类数组的数据类型。
    \item 可选的类型标注:Julia拥有丰富的数据类型描述,类型声明可以使得程序更加可读和健壮。
    \item 可组合:Julia的包可以很自然的组合运行。单位数量的矩阵或数据表一列中的货币和颜色可以一起组合使用并且拥有良好的性能。
\end{itemize}

简单来说,Julia与其他编程语言相比,有着如下的特点\cite{Julia2023}:

\begin{itemize}
    \item 核心语言很小:标准库是用 Julia 自身写的,包括整数运算这样的基础运算
    \item 丰富的基础类型:既可用于定义和描述对象,也可用于做可选的类型标注
    \item 通过多重派发,可以根据类型的不同,来调用同名函数的不同实现
    \item 为不同的参数类型,自动生成高效、专用的代码
    \item 接近 C 语言的性能
    \item 对 Unicode 的有效支持,包括但不限于 UTF-8
    \item 直接调用 C 函数,无需封装或调用特别的 API
\end{itemize}

在UVM中,引入了agent和sequence的概念,因此UVM中验证平台的典型框图如图\ref{UVM typical testbench}所示。

\begin{figure}[!htb]
	\centering%\captionsetup{font={small}} %small or scriptsize
	\tikzstyle{every node}=[on grid, align = center, font=\normalsize, node distance=1 and 2]
	\tikzstyle{rec} = [draw,thick,minimum height=0.6cm,minimum width=1.8cm]
	\scalebox{0.9}{
	\begin{tikzpicture}
		%\draw[thick,draw=gray] (-3,-5) grid (8,1);
		%\node (Transaction) [rec,color=red] {Transaction};
		\node (ref) [rec,color=blue,minimum width=3.5cm] {reference model};
		\node (sequencer) [rec,color=blue,below left=2.0 and 1.0 of ref] {sequencer};
		\node (driver) [rec,color=blue,below=of sequencer] {driver};
		\node (monitor1) [rec,color=blue,right=of driver] {monitor};

		\node (scoreboard) [rec,color=blue,minimum width=3.5cm,right=5.0 of ref] {scoreboard};
		\node (monitor2) [rec,color=blue,below right=3.0 and 6.0 of ref] {monitor};

		\node (in_agent) [rec,color=blue,below right=0.5 and 1.0 of sequencer,minimum height=2.2cm,minimum width=4.6cm] {};
		\node (out_agent) [rec,color=blue,above left=0.5 and 1.0 of monitor2,minimum height=2.2cm,minimum width=4.6cm] {};

		\node (in_agent_text) [color=blue,right=of sequencer] {in\_agent};
		\node (out_agent_text) [minimum height=0.6cm,minimum width=1.8cm,color=blue,above=of monitor2] {out\_agent};

		\node (env) [rec,color=blue,below right=1.5 and 2.5 of ref,minimum height=5.2cm,minimum width=10.8cm] {};
		\node (env_text) [color=blue,above right=2.2 and 4.0 of env] {env};
		\node (DUT) [rec,dotted,color=blue,below =3.3 of env] {DUT};

		\draw[thick,-Latex] (in_agent) -- (ref);
		\draw[thick,-Latex] (ref) -- (scoreboard);
		\draw[thick,-Latex] (out_agent) -- (scoreboard);

		\draw[thick,-Latex] (driver) |- (DUT);
		\node (DOT) [circle,draw,fill=black,scale=0.4,below =1.8 of monitor1] {};
		\draw[thick,-Latex] (DOT) -- (monitor1);
		\draw[thick,-Latex] (DUT) -| (monitor2);
	\end{tikzpicture}
	}
	\caption{简单验证平台框图}\label{UVM typical testbench}
\end{figure}

从下一节开始,将从只有一个driver的最简单的验证平台开始,一步一步搭建如图\ref{UVM typical testbench}所示的验证平台。

\section{只有driver的验证平台}\label{sec2-2}
driver是验证平台最基本的组件,是整个验证平台数据流的源泉。本节以一个简单的DUT为例,搭建一个只有driver的UVM验证平台。

\subsection{最简单的验证平台}\label{subsec2-2-1}
在本章中,假设有如下的DUT定义\footnote{如果例子的标题为文件名及路径,表明此段代码可以从本书的源码包中找到。}:

\lstinputlisting[language=SystemVerilog, caption=\lstname,numbers=left,label=code2-1]{src/ch02/dut/dut.sv}

UVM验证平台中的driver应该派生自uvm\_driver,一个简单的driver如下例所示:

\lstinputlisting[language=SystemVerilog, caption=\lstname,numbers=left,linerange={4-24},consecutivenumbers=false,label=code2-2]{src/ch02/add_driver/my_driver.sv}
