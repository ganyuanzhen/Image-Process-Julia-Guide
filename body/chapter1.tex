\chapter{Julia入门}\label{ch1}

\section{Julia简介}\label{sec1-1}

Julia 是一个非常棒的科学计算语言。它像 R、MATLAB 和 Python 一样简单,在高级数值计算方面有丰富的表现力,并且支持通用编程。根据官网的描述,Julia是一门这样的语言\cite{Juliadevelopers2018}:

\begin{itemize}
    \tightlist
	\item 快速:Julia一开始就是为高性能而设计的。Julia可以通过LLVM而跨平台被编译成高效的本地代码。
	\item 通用:Julia使用多分派作为编程范式,使其更容易表达面向对象和函数式编程范式。标准库提供了异步I/O,进程控制,日志记录,性能分析,包管理器等等。
    \item 动态:Julia是动态类型的,与脚本语言类似,并且对交互式使用具有很好的支持。
    \item 数值计算:Julia擅长于数值计算,它的语法适用于数学计算,支持多种数值类型,并且支持并行计算。Julia的多分派自然适合于定义数值和类数组的数据类型。
    \item 可选的类型标注:Julia拥有丰富的数据类型描述,类型声明可以使得程序更加可读和健壮。
    \item 可组合:Julia的包可以很自然的组合运行。单位数量的矩阵或数据表一列中的货币和颜色可以一起组合使用并且拥有良好的性能。
\end{itemize}

简单来说,Julia与其他编程语言相比,有着如下的特点\cite{Julia2023}:

\begin{itemize}
    \tightlist
    \item 核心语言很小:标准库是用 Julia 自身写的,包括整数运算这样的基础运算
    \item 丰富的基础类型:既可用于定义和描述对象,也可用于做可选的类型标注
    \item 通过多重派发,可以根据类型的不同,来调用同名函数的不同实现
    \item 为不同的参数类型,自动生成高效、专用的代码
    \item 接近 C 语言的性能
    \item 对 Unicode 的有效支持,包括但不限于 UTF-8
    \item 直接调用 C 函数,无需封装或调用特别的 API
\end{itemize}

\section{Julia一键安装}\label{sec1-2}

\href{https://julialang.org/downloads/}{Julia官网}的下载通常较慢,所以这里推荐大家从国内的镜像下载~

对于习惯命令行的用户而言,jill.py 是一个社区维护的全平台下一键安装 Julia 的命令行工具。

\begin{itemize}
    \tightlist
    \item 安装/更新 jill: pip install jill --user -U (需要 Python 3.6 或更新的版本)

    \item 安装 Julia:jill install [VERSION] [--upstream UPSTREAM] [--confirm]

    \item jill install:最新的 x.y.z 版本

    \item jill install --confirm:无需交互确认直接安装

    \item jill install --upstream BFSU:从北外镜像下载并安装

    \item jill install 1.4:安装最新的 1.4.z 版本

    \item 查询现存的上游镜像:jill upstream

    \item 帮助文档:jill [COMMAND] --help

    \item jill --help:查询存在的 jill 命令

    \item jill install --help:查询 install 命令的使用方式
\end{itemize}

利用 jill 安装完成后即可通过在命令行执行 julia/julia-1/julia-1.4 来启动不同版本的 Julia。

使用镜像站来加速下载几乎是每个国内用户都需要了解的事情,关于镜像站的使用说明及汇总可以在 \href{https://discourse.juliacn.com/t/topic/2969}{Julia PkgServer 镜像服务及镜像站索引 }中可以看到。

接下来的语法介绍基本来自于这里\cite{Julia2023}。没有基础的的读者可以先阅读这一部分~

\section{Julia语法基础}\label{sec1-3}

在本节,会简单介绍Julia的语法基础,比如变量、字符串、函数、流程控制等等。更深层次的用法请阅读Julia的文档 。

\subsection{变量}\label{sec1-3-1}

Julia 语言中,变量是与某个值相关联(或绑定)的名字。你可以用它来保存一个值(例如某些计算得到的结果),供之后的代码使用。例如:

\begin{GitExampla}{Julia@xubuntu:$\sim$}
    julia> x = 10
    10

    julia> x + 1
    11

    julia> x = 1 + 1
    2

    julia> x = "Hello World!"
    "Hello World!"
\end{GitExampla}
